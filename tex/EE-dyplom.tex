% !TEX program = xelatex
% !TeX encoding = utf8
% !TeX spellcheck = pl-PL

%%%%%%%%%%%%%%%%%%%%%%%%%%%%%%%%%%%%%%%%%%%%%%%%%%%%%%%%%%%%%%%%%%%%%%%%%%%
% Wybierz rodzaj pracy dyplomowej oraz wydział
% Pick thesis type and faculty
%%%%%%%%%%%%%%%%%%%%%%%%%%%%%%%%%%%%%%%%%%%%%%%%%%%%%%%%%%%%%%%%%%%%%%%%%%%
\documentclass[thesis=inz,faculty=ee]{EE-dyplom} 
\usepackage{amsmath}
\usepackage{amssymb}
\usepackage{bbm,graphicx} 

    %add P.H.
    %\usepackage[default]{sourcecodepro}
    %\usepackage[T1]{fontenc}

% thesis=[inz|mgr|bsc|msc]
%  * inz - praca inżynierska
%  * mgr - praca magisterska
%  * bsc - bachelor thesis
%  * msc - master thesis

% Skróty nazw wydziałów zgodne z domenami internetowymi
% Abbreviations of Faculties according to Internet subdomains
% faculty=[
%	arch,
%	gik,
%	ee,
%	wip
%	]

%%%%%%%%%%%%%%%%%%%%%%%%%%%%%%%%%%%%%%%%%%%%%%%%%%%%%%%%%%%%%%%%%%%%%%%%%%%
% Konfiguracja - do personalizacji
% Configuration - to be personalized
%%%%%%%%%%%%%%%%%%%%%%%%%%%%%%%%%%%%%%%%%%%%%%%%%%%%%%%%%%%%%%%%%%%%%%%%%%%
 \instytut{Instytut Sterowania i Elektroniki Przemysłowej}
\kierunek{Informatyka Stosowana}
%\specjalnosc{Rzeczowidztwo}
\title{Porównanie wydajności rozwiązań do realizacji sieci neuronowych w językach: Matlab, Python, C++.}
\engtitle{Comparison of the performance of solutions for implementing neural networks in: Matlab, Python, C++.}
\album{146703}
\author{Piotr Heinzelman}
\promotor{dr inz. Witold Czajewski}
\date{2024}
\longdate{2024-09-27}

%\grantlicense{TRUE} % [TRUE|FALSE]

%%%%%%%%%%%%%%%%%%%%%%%%%%%%%%%%%%%%%%%%%%%%%%%%%%%%%%%%%%%%%%%%%%%%%%%%%%%
% Streszczenie pracy i abstract.
% In case of thesis in English swap the order - English version goes first.
%%%%%%%%%%%%%%%%%%%%%%%%%%%%%%%%%%%%%%%%%%%%%%%%%%%%%%%%%%%%%%%%%%%%%%%%%%%
\streszczeniepracy{



Idea jest taka, by sprawdzić jak w różnych językach progrmowania wygląda jakieś kompleksowe zastosowanie sieci neuronowych do przetwarzania obrazów. Dwa najpopularniejsze języki to C++ i Python, potem jest Matlab. Dość powszechne jest też uruchamianie kodu Pythona w Colabie. Chodzi o zrobienie dokładnie kilku takich samych aplikacji w kilku językach i porównanie ich pod różnymi kątami. Przykładowe aplikacje to: klasyfikacja obrazów, rozpoznawanie obiektów na obrazach, śledzenie obiektów, segmentacja obiektów, modyfikacja obrazów czy ich fragmentów.

Jeśli taki zakres projektu Panom odpowiada, to proszę dać znać, a założę dedykowany projektowi kanał na Teamsach, gdzie będziemy się dalej komunikować.

} % koniec streszczenia

\slowakluczowe{A, B, C}

\thesisabstract{
This is abstract. This one is a little too short as it should occupy the whole page.

\lipsum[1-4]
} % end of abstract

\thesiskeywords{X, Y, Z}

%%%%%%%%%%%%%%%%%%%%%%%%%%%%%%%%%%%%%%%%%%%%%%%%%%%%%%%%%%%%%%%%%%%%%%%%%%%
% Tu zaczyna się dokument
% Here is the beginning of the document
%%%%%%%%%%%%%%%%%%%%%%%%%%%%%%%%%%%%%%%%%%%%%%%%%%%%%%%%%%%%%%%%%%%%%%%%%%%
\begin{document}
    % Strony nagłówkowe
    % Headers
    \frontpages

    % Właściwa treść jest w pliku tekst/main.tex
    % Real contents is in tekst/main.tex
    % Rozdziały zaczynają się od "chapter"
\chapter{Wstęp}
% Praca podzielona na mniejsze pliki włączane za pomocą input
% Zajrzyj do pliku tekst/wstep.tex
Dobór algorytmu do zadania jest bardzo ważny, zdecydowanie ważniejszy niż dobór języka - jednak nie będzie on głównym tematem tej pracy. Tu zakładamy użycie tych samych algorytmów i porównujemy wydajności implementacji algorytmu w różnych "językach". Teoretycznie wyniki powinny być zbliżone przy założeniu, że programy doskonale wykorzystują możliwości sprzętu. Celem pracy jest porównanie, które języki ogólnego przeznaczenia liczą szybciej, które lepiej wykorzystują dodatkową infrastrukturę taką jak \textit{wątki}, \textit{procesy},  \textit{hyperThreading}, czy rdzenie \textit{CUDA} na kartach graficznych.
Jeśli gdzieś zaobserwujemy różnice - to będą one wynikały zastosowania innych języków, które w odmienny sposób zapisują i odczytują liczby, kolejkują zadania czy optymalizują wygenerowany kod. \newLine 
Jednak zanim przejdziemy do~porównania, przyjrzymy się modelom matematycznym, a z ich pomocą zbudujemy prosty model obiektowy. Następnie przetestujemy model obiektowy, prześledzimy przetwarzanie na najniższych poziomach, aż do pojedynczych operacji. Te działania pomogą nam przygotować zadania numeryczne, do rozwiązania których użyjemy maszyn cyfrowych.

\begin{lstlisting}
/*

4. Realizacje obliczeń Klasyfikowanie pisma odręcznego MLP z wykorzystaniem danych MNIST
a) Matlab
b) Python -numpy, -sklearn
c) Python -tensorflow
d) własna implementacja w Java

Cel podstawowy: porównanie wydajności realizacji w zależności od Języka.
Cele dodatkowe: potwierdzenie poprawności własnej implementacji przez porównanie wyników,


5) realizacja Klasyfikowanie pisma odręcznego sieci głębokie CNN realizacja sieci z wykorzytsniem bibliotek.
a) Matlab
b) Python tensorflow.keras
c) własna implementacja Java lub C++ libtorch

Cel podstawowy: porównanie wydajności implementacji. Zbadanie wydajności uczenia sieci.
Cele dodatkowe: potwierdzenie poprawności własnej implementacji w Java lub C++


6) Rozpoznawanie twarzy sieci głębokie CNN
a) Matlab
b) Python tensorflow.keras
c) własna implementacja Java lub C++ libtorch

Cel podstawowy: porównanie wydajności implementacji.


7) Klasyfikacji obrazów z wykorzystaniem sieci głębokiej CNN
a) Matlab
b) Python tensorflow.keras
c) C++ libtorch

Cel podstawowy: porównanie wydajności implementacji i wydajności procesu uczenia nadzorowanego.

8) detekcja i segmentacji obiektów sieci głębokiej CNN
b) Python tensorflow.keras
c) C++ libtorch

Cel podstawowy: porównanie wydajności implementacji.


*/
\end{lstlisting}

% Można też wszystko pisać w jednym pliku ale będzie on duży
\chapter{Nienudny tytuł dla teorii}
Można też pisać wszystko w~jednym pliku, tak jak przyzwyczajają do tego gorsze programy, ale wtedy główny plik będzie bardzo duży i~trudniejszy w zarządzaniu.

I~to by było na tyle. Kolejny rozdział jest testem ciągłości numeracji rysunków, wzorów i~innych elementów graficznych. Za nim jest jeszcze rozdział~\ref{ch:podsumowanie} z podsumowaniem, bibliografia, wykazy, spisy i załączniki.

% fragment nieużywany albo jeszcze niedodany można zakomentować
%\input{tekst/teoria}
%\input{tekst/donapisania}

\chapter{Niebanalny tytuł kolejnego rozdziału}

\chapter{ Jednowymiarowa regresja liniowa}
\cite{russell2023} Funkcja liniowa jednej zmiennej to funkcja w postaci \(y=w_1x +w_0\); współczynnki \(w_0\) i \(w_1\) możemy traktować jak wagi, i możemy je traktować łącznie jako wektor \(\textbf{W}=<w_0,w_1>\) a samo przekształcenie można utożsamić z iloczynem skalarnym \(y=\textbf{W}*<1,x> \). Zadanie dopasowania najlepszej hipotezy \(hw\) wiążącej te dwie wielkości nosi nazwę regresji liniowej. Matematycznie dopasowanie to sprowadza się do znalezienia wektora W minimalizującego funkcję straty, zgodnie z teorią Gaussa jako miarę tej straty przyjmuje się sumę miar dla wszystkich przykładów:
\begin{equation}
       Loss(h_w) = \sum_{j=1}^{N} L_2 \big{(}y_j, hw(x_j)\big{)} = \sum_{j=1}^{N} L_2 \big{(}y_j-hw(x_j)\big{)}^2 =
       \sum_{j=1}^{N} L_2 \big{(}y_j-(w_1x + w_0)\big{)}^2,
\end{equation}
Naszym celem jest znalezienie optymalnego wektora W 
\begin{equation}
       \textbf{W} = \text{argmin } Loss(h_w)
\end{equation}
Gdy funkcja ciągła osiąga minimum w danym punkcie, pierwsze pochodne cząstkowe po argumentach tej funkcji zerują się w tym punkcie; w kontekście regresji liniowej nasza funkcja \(Loss(h_w)\) jest funkcją dwu zmiennych: \(w_0\) i \(w_1\), których wartości w punkcie minimum określone są przez układ równań:
 

    \begin{equation}
        \Biggl\{
                \begin{matrix}
                    \frac{\partial}{\partial{w_0}} \sum_ \big{(}y_j -(w_1x + w_0)\big{)}^2 = 0,\\
                     
                    \frac{\partial}{\partial{w_1}} \sum \big{(}y_j -(w_1x + w_0)\big{)}^2 = 0,
                \end{matrix} 
    \end{equation} 

Rozwiązaniem takiego układu są wartości:

\begin{equation}
w_1=\frac{ N(\sum x_jy_j)-(\sum x_j)(\sum y_j) }{ N(x_j^2)-(\sum x_j)^2 }, 
w_0=\frac{\sum y_j - w_1( \sum x_j)}{N},
\end{equation}

Dla dużych N\cite{russell2023} musimy użyć następującej, równoważnej postaci rzeczonych wzorów:
\begin{equation}
w_1=\frac{ \sum (x_j-\overline{x})( y_j - \overline{y}  )  }{ \sum ( x_j - \overline{x} )^2 }, 
w_0=\overline y - w_1\overline x,
\end{equation}
gdzie \(\overline{x}\) i \(\overline{y} \) są średnimi arytmetycznymi: 
\begin{equation}
\overline{x}=\frac{\sum x_j}{N}, \overline{y}=\frac{\sum y_j}{N},
\end{equation}


\subsection{Realizacja obliczeń}
Przykłady obliczeń Python i Matlab zostały zaczerpnięte z [ossowski2023].
pełen kod dostępny pod adresem: https://github.com/piotrHeinzelman/inz/tree/main
w przypadku Matlab i Python korzystam z dostępnych funkcji, w przypadku Java obliczam wg. wzoru ! 23 !. Obliczenia różnymi metodami dają zbliżone wyniki, więc zakładam że moje implementacje są poprawne. 

Matlab: 
\begin{lstlisting}

    x=[ 1 2 3 4 5 6 7 8 9 10 11 12 13 14 15 16 17 18 19 20 ]
    y=[ -1.69 -0.79 5.77 7.80 4.56 14.32 15.47 8.88 7.41 17.26 14.83 20.47 20.39 27.04 22.53 22.36 29.35 22.86 31.22 28.13 ]

    for i = 1:n
        a = polyfit(x,y,1);
    end
\end{lstlisting}

Python: 
\begin{lstlisting}
for i in range( cycles ):
    a = np.polyfit(x,y,1)
\end{lstlisting}

Java:
\begin{lstlisting}
for ( int C=0; C<cycles; C++ ) {
        double xsr = 0.0;
        double ysr = 0.0;
        for (int i = 0; i < x.length; i++) {
            xsr += x[i];
            ysr += y[i];
        }
        xsr = xsr / x.length;
        ysr = ysr / y.length;

        w1 = 0.0;
        w0 = 0.0;
        double sumTop = 0.0;
        double sumBottom = 0.0;
        for (int i = 0; i < x.length; i++) {
            sumTop += ((x[i] - xsr) * (y[i] - ysr));
            sumBottom += ((x[i] - xsr) * (x[i] - xsr));
        }
        w1 = sumTop / sumBottom;
        w0 = ysr - w1 * xsr;
    }
\end{lstlisting}

czasy wykonania kodu polyfit 20 próbek: 
Java:    0.042  sek. !!!    
Matlab:  5.936 sek.
Python: 26.918 sek.





\section{Spadek gradientowy} Jako że naszym celem jest minimalizowanie straty, rozpoczynamy od dowolnego punktu na płaszczyźnie \(<w0,w1>\), wyliczamy przybliżenie gradientu w tym punkcie, i czynimy niewielki krok w kierunku wyznaczonym przez gradient. W przypadku regresji jednowymiarowej funkcja straty jest funkcją kwadratową, więc pochodne cząstkowe są funkcjami liniowymi. 
\begin{equation}
\begin{split}
\frac {\partial g ( f(x) )}{\partial x} = \frac {g' ( f(x) ) \partial f(x)}{\partial x},\\
\\
\frac {\partial}{\partial x}x^2=2x, \text{ i }
\frac {\partial}{\partial x}x=1,
\end{split}
\end{equation}

\begin{equation}
\begin{split}
\frac {\partial}{ \partial w_i} Loss(w) = 
\frac {\partial}{ \partial w_i} (y-h_w(x))^2 =2(y-h_w(x)) * \frac{\partial}{\partial w_i}(y-h_w(x)) = \\ = 2(y-h_w(x))*\frac{\partial}{\partial w_i}(y-(w_1x + w_0))
\end{split}
\end{equation}


Pochodne cząstkowe dla parametrów:
\begin{equation}
\begin{split}
\frac {\partial}{ \partial w_0} Loss(w) = -2(y-h_w(x)), \\
\frac {\partial}{ \partial w_1} Loss(w) = -2(y-h_w(x))*x, \\
\end{split}
\end{equation}
Reguła uczenia wag: gdzie \(\alpha\) jest stałą uczenia, około 0,01
\begin{equation}
\begin{split}
w_0 \leftarrow w_0 + \alpha \sum (y_j - h_w(x_i)),\\
w_1 \leftarrow w_1 + \alpha \sum (y_j - h_w(x_i))*x_i,\\
\end{split}
\end{equation}
Dla wielozmiennej regresji liniowej mamy \cite{russell2023} 
\begin{equation}
\begin{split}
w_i \leftarrow w_i + \alpha \sum_{j} \big{(}y_j - h_w(x_j))*x_j,_i\big{)},\\
\end{split}
\end{equation}
A błąd kwadratowy dla całego zbioru: \cite{russell2023} 
\begin{equation}
L(W)=||\overline y-y||^2 = ||XW-Y||^2
\end{equation}
\section{Obliczenia - Matlab} 
Rozwiązanie \cite{ossowski2023} możemy zaimplementować bezpośrednio w Matlabie, albo wykorzystać funkcję  \textit{ polyfit }  przy założeniu stopnia wielomianu równego jeden.




\chapter{ SVN - maszyny wektorów nośnych }
\cite{russell2023} Na początkuXXI wieku popularność zyskała sobie klasa modeli o nazwie \textbf{maszyny wektorów nośnych} (ang. \textit{Support Vector Machine}) stanowiąca proste podejście do nadzorowanego uczenia. 

 
--------------


\begin{figure}[!hb]
	\centering \includegraphics[width=0.618\linewidth]{Kopernik.jpg}
	\caption{Powtórzony rysunek dla testu ciągłości numeracji}
	\label{rys:kopernik2}
\end{figure}

\begin{table}[!b]
 \centering
  \begin{tabular}{p{2.5cm}c|l}
    Data        &   Godzina (UTC)   &   Zdarzenie\\\hline
    2016-05-09  &   14:57           &   Tranzyt Merkurego\\\hline
    2017-08-11 --~2017-08-13  & --- &   Maksimum Perseidów \\\hline
    2018-07-27  &   20:22           &   Całkowite zaćmienie Księżyca\\\hline
    2019-08-24  &   17:04           &   Koniunkcja Wenus i Mars w odległości - 0°17`\\\hline
    2020-12-21  &   16:00           &   Koniunkcja Jowisz i Saturn w odległości 0°06`
  \end{tabular}
 \caption{\label{tab:zjawiska2}Powtórzona tabelka dla testu ciągłości numeracji}
\end{table}

\begin{equation}
    \frac{\partial^2 y}{\partial x^2} = \frac{\mu}{F} \; \frac{\partial^2 y}{\partial t^2}
\end{equation}

\begin{lstlisting}[language=Python,
    caption={Powtórzony kod dla testu ciągłości numeracji},
    label={lst:hello2}]
#!/usr/bin/env python
# -*- coding: utf-8 -*-
"""Simple world of hello.
"""

import sys

def main():
    """The one and only function"""
    fib = lambda n: reduce(lambda x, n: [x[1], x[0]+x[1]], range(n), [0, 1])[0]
    try:
        print(fib(int(sys.argv[1])))
    except:
        print("Hello World!")

if __name__ == "__main__":
    main()
\end{lstlisting}


% Przykładowy wypełniacz
\bredzenie{21-40}


\chapter{Podsumowanie}
\label{ch:podsumowanie}
\begin{flushright}
\framebox{%
 \begin{minipage}{0.75\linewidth}
  \begin{itemize}
    \item[--] Teoria: wiemy jak ma działać, ale jednak nie działa.
    \item[--] Praktyka: działa, ale nie wiemy -- dlaczego?
    \item[--] Łączymy teorię z praktyką: nic nie działa i nie wiemy, dlaczego.
  \end{itemize}
  \end{minipage}
 }
\end{flushright}

Więcej informacji na temat \LaTeX{a}:
\begin{itemize}
    \item \href{https://www.overleaf.com/learn}{<https://www.overleaf.com/learn>} -- przystępny tutorial na stronie Overleaf,
    \item \href{https://www.latex-project.org/}{<https://www.latex-project.org/>} -- strona domowa projektu,
    \item \href{https://www.tug.org/begin.html}{<https://www.tug.org/begin.html>} -- dobry zbiór odnośników do innych materiałów.
\end{itemize}

\vspace{2ex}
\begin{flushright}
 Powodzenia!
\end{flushright}



    % Bibliografia - musi być
    % Bibliography - must exist
    \bibliografia

    % Strony końcowe - można zakomentować, jeśli zbędne
    % Additional pages - comment out if not needed
    
    % Wykaz symboli i skrótów - patrz opis w tekście przykładowym
    \acronymslist
    % Spis rysunków
    \listoffigures
    % Spis tabel
    \listoftables
    % Załączniki (plik appendices.tex)
    \easyappendices
\end{document}
%%%%%%%%%%%%%%%%%%%%%%%%%%%%%%%%%%%%%%%%%%%%%%%%%%%%%%%%%%%%%%%%%%%%%%%%%%%

