"Sztuczna inteligencja" zyskała w ostatnim czasie ogromną popularność. Wielkie firmy technologiczne udostępniły użytkownikom możliwość komunikacji ze swoimi wielkimi modelami językowymi LLM. Powstały modele językowe pracujące w języku polskim, m.in "Bielik", na dziś dostępna jest wersja BIELIK-11B-v2 mająca 11 000 000 parametrów. Uczenie tak dużych modeli szybciej niż konkurencja wymaga posiadania wielkich centrów obliczeniowych. 


W cieniu modeli językowych pozostają inne, dużo mniej medialne rozwiązania wykorzystywane od dziesięcioleci w przemyśle modele sieci MLP \cite{Korbicz1994},  Dziedzina "widzenia komputerowego" dostarcza modeli sieci głębokich CNN używanych w rozpoznawaniu obrazów, ich klasyfikacji a także  orientacji w przestrzeni. Modele te są używane zwłaszcza w robotyce. 
\cite{Osowski2023},
 \cite{Osowski2020}.
 \cite{rasheed},

 \cite{conv},
\cite{conv1}.